\documentclass{article}

\usepackage{hyperref}

\author{}
\date{\today}
\title{Challenge Tracker - Project Report}
\begin{document}
	\maketitle
	\section{Concept}
	\subsection{Idea}
	The general idea is to create an application that eases team-challenges. Challenges consist of a mapping of several disciplines (e.g. running, biking, ...) to corresponding points per km, a defined goal how many points should be collected and a deadline. Users can use the app for tracking their activities which automatically are stored in a database and see the current status of the challenge.
	\subsection{Features and Requirements}
	\label{sec:req}
	The application has to be able to create and track challenges. It must be possible to see the current status of a challenge and add new activities through tracking movement via GPS.
	\subsection{Audience and Use Case}
	The intended user for this application are people from a team. Together, they track their activities and contribute to common challenges. 
	
	\section{Task Fulfillment}
	Core Blocks
	\begin{itemize}
		\item Multiple screens (min 3)
		\item Settings Screen
		\item No crash, Lifecycle, Permissions
		\item Ressource files
		\item Background workers
	\end{itemize}
	
	The feature Blocks
	\begin{itemize}
		\item GPS
		\item Persistent Storage
		\item Services
	\end{itemize}
	are used and thus, the requirement on the feature blocks is fulfilled. A GPS service is implemented to track activities. Firebase is used as a persistent storage in order to be able to share the activities and challenges between the users. 
	
	The own requirements from section \ref{sec:req} are fulfilled.
	\section{Work}
	The project was distributed into three parts:
	\begin{itemize}
		\item DataBase (Navid)
		\item GPS (Ulrike)
		\item Settings (Karl)
	\end{itemize}
	During the developement, each member of the team worked on his/her main task. In order to keep a common idea, a weekly meeting was instantiated. During the whole projects, minor questions and bugs were reported in a chat group.
	
	change of idea
	Problems
	\section{Future}
	Multi Platform
	
	Manual entries, deletion
	
	Store activity details(path) locally
	
	Compare to other teams
	
	\section{Link to repository}
	\url{https://github.com/navid-bamdad-roshan/challenge-tracker}
	
	

\end{document}